% Generated by Sphinx.
\def\sphinxdocclass{report}
\documentclass[letterpaper,10pt,english]{sphinxmanual}
\usepackage[utf8]{inputenc}
\DeclareUnicodeCharacter{00A0}{\nobreakspace}
\usepackage[T1]{fontenc}
\usepackage{babel}
\usepackage{times}
\usepackage[Bjarne]{fncychap}
\usepackage{longtable}
\usepackage{sphinx}
\usepackage{multirow}


\title{KMC Documentation}
\date{July 15, 2014}
\release{1.0}
\author{David Abbasi}
\newcommand{\sphinxlogo}{}
\renewcommand{\releasename}{Release}
\makeindex

\makeatletter
\def\PYG@reset{\let\PYG@it=\relax \let\PYG@bf=\relax%
    \let\PYG@ul=\relax \let\PYG@tc=\relax%
    \let\PYG@bc=\relax \let\PYG@ff=\relax}
\def\PYG@tok#1{\csname PYG@tok@#1\endcsname}
\def\PYG@toks#1+{\ifx\relax#1\empty\else%
    \PYG@tok{#1}\expandafter\PYG@toks\fi}
\def\PYG@do#1{\PYG@bc{\PYG@tc{\PYG@ul{%
    \PYG@it{\PYG@bf{\PYG@ff{#1}}}}}}}
\def\PYG#1#2{\PYG@reset\PYG@toks#1+\relax+\PYG@do{#2}}

\def\PYG@tok@gd{\def\PYG@tc##1{\textcolor[rgb]{0.63,0.00,0.00}{##1}}}
\def\PYG@tok@gu{\let\PYG@bf=\textbf\def\PYG@tc##1{\textcolor[rgb]{0.50,0.00,0.50}{##1}}}
\def\PYG@tok@gt{\def\PYG@tc##1{\textcolor[rgb]{0.00,0.25,0.82}{##1}}}
\def\PYG@tok@gs{\let\PYG@bf=\textbf}
\def\PYG@tok@gr{\def\PYG@tc##1{\textcolor[rgb]{1.00,0.00,0.00}{##1}}}
\def\PYG@tok@cm{\let\PYG@it=\textit\def\PYG@tc##1{\textcolor[rgb]{0.25,0.50,0.56}{##1}}}
\def\PYG@tok@vg{\def\PYG@tc##1{\textcolor[rgb]{0.73,0.38,0.84}{##1}}}
\def\PYG@tok@m{\def\PYG@tc##1{\textcolor[rgb]{0.13,0.50,0.31}{##1}}}
\def\PYG@tok@mh{\def\PYG@tc##1{\textcolor[rgb]{0.13,0.50,0.31}{##1}}}
\def\PYG@tok@cs{\def\PYG@tc##1{\textcolor[rgb]{0.25,0.50,0.56}{##1}}\def\PYG@bc##1{\colorbox[rgb]{1.00,0.94,0.94}{##1}}}
\def\PYG@tok@ge{\let\PYG@it=\textit}
\def\PYG@tok@vc{\def\PYG@tc##1{\textcolor[rgb]{0.73,0.38,0.84}{##1}}}
\def\PYG@tok@il{\def\PYG@tc##1{\textcolor[rgb]{0.13,0.50,0.31}{##1}}}
\def\PYG@tok@go{\def\PYG@tc##1{\textcolor[rgb]{0.19,0.19,0.19}{##1}}}
\def\PYG@tok@cp{\def\PYG@tc##1{\textcolor[rgb]{0.00,0.44,0.13}{##1}}}
\def\PYG@tok@gi{\def\PYG@tc##1{\textcolor[rgb]{0.00,0.63,0.00}{##1}}}
\def\PYG@tok@gh{\let\PYG@bf=\textbf\def\PYG@tc##1{\textcolor[rgb]{0.00,0.00,0.50}{##1}}}
\def\PYG@tok@ni{\let\PYG@bf=\textbf\def\PYG@tc##1{\textcolor[rgb]{0.84,0.33,0.22}{##1}}}
\def\PYG@tok@nl{\let\PYG@bf=\textbf\def\PYG@tc##1{\textcolor[rgb]{0.00,0.13,0.44}{##1}}}
\def\PYG@tok@nn{\let\PYG@bf=\textbf\def\PYG@tc##1{\textcolor[rgb]{0.05,0.52,0.71}{##1}}}
\def\PYG@tok@no{\def\PYG@tc##1{\textcolor[rgb]{0.38,0.68,0.84}{##1}}}
\def\PYG@tok@na{\def\PYG@tc##1{\textcolor[rgb]{0.25,0.44,0.63}{##1}}}
\def\PYG@tok@nb{\def\PYG@tc##1{\textcolor[rgb]{0.00,0.44,0.13}{##1}}}
\def\PYG@tok@nc{\let\PYG@bf=\textbf\def\PYG@tc##1{\textcolor[rgb]{0.05,0.52,0.71}{##1}}}
\def\PYG@tok@nd{\let\PYG@bf=\textbf\def\PYG@tc##1{\textcolor[rgb]{0.33,0.33,0.33}{##1}}}
\def\PYG@tok@ne{\def\PYG@tc##1{\textcolor[rgb]{0.00,0.44,0.13}{##1}}}
\def\PYG@tok@nf{\def\PYG@tc##1{\textcolor[rgb]{0.02,0.16,0.49}{##1}}}
\def\PYG@tok@si{\let\PYG@it=\textit\def\PYG@tc##1{\textcolor[rgb]{0.44,0.63,0.82}{##1}}}
\def\PYG@tok@s2{\def\PYG@tc##1{\textcolor[rgb]{0.25,0.44,0.63}{##1}}}
\def\PYG@tok@vi{\def\PYG@tc##1{\textcolor[rgb]{0.73,0.38,0.84}{##1}}}
\def\PYG@tok@nt{\let\PYG@bf=\textbf\def\PYG@tc##1{\textcolor[rgb]{0.02,0.16,0.45}{##1}}}
\def\PYG@tok@nv{\def\PYG@tc##1{\textcolor[rgb]{0.73,0.38,0.84}{##1}}}
\def\PYG@tok@s1{\def\PYG@tc##1{\textcolor[rgb]{0.25,0.44,0.63}{##1}}}
\def\PYG@tok@gp{\let\PYG@bf=\textbf\def\PYG@tc##1{\textcolor[rgb]{0.78,0.36,0.04}{##1}}}
\def\PYG@tok@sh{\def\PYG@tc##1{\textcolor[rgb]{0.25,0.44,0.63}{##1}}}
\def\PYG@tok@ow{\let\PYG@bf=\textbf\def\PYG@tc##1{\textcolor[rgb]{0.00,0.44,0.13}{##1}}}
\def\PYG@tok@sx{\def\PYG@tc##1{\textcolor[rgb]{0.78,0.36,0.04}{##1}}}
\def\PYG@tok@bp{\def\PYG@tc##1{\textcolor[rgb]{0.00,0.44,0.13}{##1}}}
\def\PYG@tok@c1{\let\PYG@it=\textit\def\PYG@tc##1{\textcolor[rgb]{0.25,0.50,0.56}{##1}}}
\def\PYG@tok@kc{\let\PYG@bf=\textbf\def\PYG@tc##1{\textcolor[rgb]{0.00,0.44,0.13}{##1}}}
\def\PYG@tok@c{\let\PYG@it=\textit\def\PYG@tc##1{\textcolor[rgb]{0.25,0.50,0.56}{##1}}}
\def\PYG@tok@mf{\def\PYG@tc##1{\textcolor[rgb]{0.13,0.50,0.31}{##1}}}
\def\PYG@tok@err{\def\PYG@bc##1{\fcolorbox[rgb]{1.00,0.00,0.00}{1,1,1}{##1}}}
\def\PYG@tok@kd{\let\PYG@bf=\textbf\def\PYG@tc##1{\textcolor[rgb]{0.00,0.44,0.13}{##1}}}
\def\PYG@tok@ss{\def\PYG@tc##1{\textcolor[rgb]{0.32,0.47,0.09}{##1}}}
\def\PYG@tok@sr{\def\PYG@tc##1{\textcolor[rgb]{0.14,0.33,0.53}{##1}}}
\def\PYG@tok@mo{\def\PYG@tc##1{\textcolor[rgb]{0.13,0.50,0.31}{##1}}}
\def\PYG@tok@mi{\def\PYG@tc##1{\textcolor[rgb]{0.13,0.50,0.31}{##1}}}
\def\PYG@tok@kn{\let\PYG@bf=\textbf\def\PYG@tc##1{\textcolor[rgb]{0.00,0.44,0.13}{##1}}}
\def\PYG@tok@o{\def\PYG@tc##1{\textcolor[rgb]{0.40,0.40,0.40}{##1}}}
\def\PYG@tok@kr{\let\PYG@bf=\textbf\def\PYG@tc##1{\textcolor[rgb]{0.00,0.44,0.13}{##1}}}
\def\PYG@tok@s{\def\PYG@tc##1{\textcolor[rgb]{0.25,0.44,0.63}{##1}}}
\def\PYG@tok@kp{\def\PYG@tc##1{\textcolor[rgb]{0.00,0.44,0.13}{##1}}}
\def\PYG@tok@w{\def\PYG@tc##1{\textcolor[rgb]{0.73,0.73,0.73}{##1}}}
\def\PYG@tok@kt{\def\PYG@tc##1{\textcolor[rgb]{0.56,0.13,0.00}{##1}}}
\def\PYG@tok@sc{\def\PYG@tc##1{\textcolor[rgb]{0.25,0.44,0.63}{##1}}}
\def\PYG@tok@sb{\def\PYG@tc##1{\textcolor[rgb]{0.25,0.44,0.63}{##1}}}
\def\PYG@tok@k{\let\PYG@bf=\textbf\def\PYG@tc##1{\textcolor[rgb]{0.00,0.44,0.13}{##1}}}
\def\PYG@tok@se{\let\PYG@bf=\textbf\def\PYG@tc##1{\textcolor[rgb]{0.25,0.44,0.63}{##1}}}
\def\PYG@tok@sd{\let\PYG@it=\textit\def\PYG@tc##1{\textcolor[rgb]{0.25,0.44,0.63}{##1}}}

\def\PYGZbs{\char`\\}
\def\PYGZus{\char`\_}
\def\PYGZob{\char`\{}
\def\PYGZcb{\char`\}}
\def\PYGZca{\char`\^}
\def\PYGZsh{\char`\#}
\def\PYGZpc{\char`\%}
\def\PYGZdl{\char`\$}
\def\PYGZti{\char`\~}
% for compatibility with earlier versions
\def\PYGZat{@}
\def\PYGZlb{[}
\def\PYGZrb{]}
\makeatother

\begin{document}

\maketitle
\tableofcontents
\phantomsection\label{index::doc}


Contents:


\chapter{Introduction}
\label{intro:introduction}\label{intro:welcome-to-kmc-s-documentation}\label{intro::doc}
Lorem ipsum dolor sit amet, consectetur adipiscing elit. Etiam ac tortor
lobortis, commodo est sit amet, vehicula nibh. Sed magna justo, adipiscing eget
ante sit amet, tristique cursus enim. Donec dictum quam nunc, eu rhoncus nibh
placerat mollis. Vestibulum vitae dignissim velit. Curabitur diam eros,
dignissim ut eleifend in, tincidunt et lacus. Phasellus pharetra sapien a
pellentesque tempus. Phasellus ullamcorper, risus a euismod dapibus, diam velit
hendrerit eros, nec feugiat risus ipsum ut quam. Aenean varius eu diam sodales
vulputate. Nunc vulputate felis ante, nec rutrum ligula volutpat pellentesque.
Pellentesque habitant morbi tristique senectus et netus et malesuada fames ac
turpis egestas. Curabitur pellentesque vestibulum scelerisque. Mauris lobortis
euismod purus a adipiscing. Vestibulum aliquam ornare massa, in aliquam lectus
fringilla at. Nullam et metus non est suscipit vulputate quis quis nisi.

Quisque quis venenatis metus. Nullam iaculis placerat neque, at pulvinar ligula
ultricies eu. Fusce augue purus, mollis sit amet lacus in, interdum sagittis
nisl. Vestibulum ante ipsum primis in faucibus orci luctus et ultrices posuere
cubilia Curae; Pellentesque venenatis, ligula vitae porttitor fringilla, urna
mi convallis odio, eget varius elit lacus eget sapien. Pellentesque habitant
morbi tristique senectus et netus et malesuada fames ac turpis egestas. Cras
sed faucibus dolor. Integer fringilla, odio vitae adipiscing ultricies, eros
nulla cursus dui, vel commodo ante nibh at lectus. Maecenas vulputate dictum
fringilla. In nibh sem, scelerisque vitae condimentum eu, hendrerit ac nibh. In
hac habitasse platea dictumst.

Maecenas nec mi mauris. In hac habitasse platea dictumst. Nulla quis pulvinar
ante. Praesent quis mi in lacus molestie fringilla non eget augue. Ut aliquam
nulla eu auctor suscipit. Curabitur aliquet felis in enim auctor, malesuada
ultricies orci varius. Pellentesque habitant morbi tristique senectus et netus
et malesuada fames ac turpis egestas. Fusce molestie, urna pellentesque
faucibus viverra, augue libero viverra eros, vel bibendum sem dui in lectus.
Vivamus commodo purus vitae arcu lacinia, et accumsan nisi volutpat.

Etiam suscipit semper metus, a molestie metus aliquet condimentum. Maecenas
congue varius rhoncus. Nam a orci aliquet, venenatis orci non, vestibulum diam.
Suspendisse lobortis odio nec eros tincidunt consectetur. Maecenas sollicitudin
facilisis ultrices. Maecenas fringilla ut tellus varius venenatis. Sed accumsan
augue id felis varius facilisis. Pellentesque cursus a ligula id varius. Cras
pharetra nisl enim, et eleifend eros pulvinar a. Nunc a luctus dolor. Curabitur
laoreet justo rutrum, sagittis augue non, congue tortor. Pellentesque quis
ipsum ac nulla vestibulum facilisis nec non libero. Suspendisse venenatis neque
magna, a aliquam risus volutpat a. Ut in eleifend orci.

Sed pharetra leo eget libero aliquet aliquet. Nam sagittis nisl at tellus
posuere, in commodo orci malesuada. Donec aliquet, metus sit amet mollis
dictum, massa diam adipiscing lacus, sit amet egestas libero lacus quis ligula.
Vivamus convallis sapien at quam adipiscing euismod eget at justo. In eu
malesuada mi. Duis mattis sapien ut molestie consequat. Sed metus libero,
eleifend nec enim sed, iaculis dapibus leo. Nunc hendrerit risus rhoncus rutrum
scelerisque. Donec ornare aliquam enim dictum eleifend. Nulla facilisi.
Maecenas eleifend volutpat urna at ultricies. Pellentesque facilisis
scelerisque libero, molestie ultricies leo gravida nec. Nunc ut consectetur
ante. Sed lacinia nunc nisl, at eleifend odio hendrerit quis.  i


\chapter{Keywords}
\label{keywords:keywords}\label{keywords::doc}

\section{How to read it}
\label{keywords:how-to-read-it}
\textbf{Keywords} are writen in bold. If the keyword needs a parameter, this
parameter is written with \code{this font}. The kind of the parameter is defined
after the point, where \code{.int} means that it is an integer, \code{.real} a real
number, or \code{.str} a string. Sometimes, when the keyword is relative to a
logival variable, the presence of the keyword switchs its default value. If
the parameter is {[}between brackets{]} it means that the parameter is optional.
The (default value) is given in parentesis.  If the keyword is preceded with
an exclamation sign means that this keyword is mandatory to be included in the
input file.


\section{Keywords}
\label{keywords:id1}

\subsection{Mandatory keywords}
\label{keywords:mandatory-keywords}\begin{itemize}
\item {} 
! \textbf{dimension} \code{dimension.int}

\end{itemize}

Set the size of the supercell (dimension.int x dimension.int). It must be
bigger than 2
\begin{itemize}
\item {} 
! \textbf{deposition\_rate} \code{deposition.real}

\end{itemize}

Set the deposition rate in ps $^{\text{-1}}$
\begin{itemize}
\item {} 
! \textbf{coverage} \code{coverage.real}  (0-1{]}

\end{itemize}

Set the coverage of the surface. Values range from 0 to 1.
\begin{itemize}
\item {} 
! \textbf{number\_of\_kmc\_steps} \code{steps.int}

\end{itemize}

Set the total number of steps of the simulation.
\begin{itemize}
\item {} 
! \textbf{barriers} 6 x \code{barriers.real}

\end{itemize}

Set the energy barriers in eV. Descriptions of the barriers are shown in the table
below.
\begin{quote}

\begin{tabulary}{\linewidth}{|L|L|}
\hline
\textbf{
\#
} & \textbf{
Description
}\\\hline

1
 & 
PTMDC diffusion via pivoting mechanism
\\\hline

2
 & 
PTMDC-PTMDC detachement (a dimer); then additive
\\\hline

3
 & 
PTMDC single isomerization
\\\hline

4
 & 
PTMDC extra barrier for assisted isomerization
\\\hline

5
 & 
PTMDC desorption from surface
\\\hline

6
 & 
PTMDC extra isomerization barrier for trans monomers
\\\hline
\end{tabulary}

\end{quote}
\begin{itemize}
\item {} 
! \textbf{temperature} \code{temperature.real}

\end{itemize}

Set the temperature of the simulation in Kelvin.


\subsection{Detailing the simulations}
\label{keywords:detailing-the-simulations}\begin{itemize}
\item {} 
\textbf{ignore\_single\_HB}

\end{itemize}

If present, single hydrogen bond will not be taken into account.
\begin{itemize}
\item {} 
\textbf{seed} \code{seed.int}

\end{itemize}

Choose the seed for the random number generation. If seed \textgreater{} 0 will use the
given seed. If seed \textless{}= 0 or not given, seed will be choose randomly.
\begin{itemize}
\item {} 
\textbf{do\_testing}

\end{itemize}

If present, the program enters in the testing mode.
\begin{itemize}
\item {} 
\textbf{restart} \code{restartfile.str}

\end{itemize}

If present, KMC will restart from a previous calculation. The file
\code{restartfile.str}  must be given and must exist.

\emph{NOTE:} In the testing mode can be generated an initial configuration of
molecules, save that configuration to a file (option -2 in the testing mode)
and choose that configuration as a restart file for run a entire simulation.


\subsection{Printing and drawing keywords}
\label{keywords:printing-and-drawing-keywords}\begin{itemize}
\item {} 
\textbf{printing} \code{printing.int} {[}0-5{]} (defaulf = 0)

\end{itemize}
\begin{quote}

Set the level of print
\end{quote}
\begin{itemize}
\item {} 
\textbf{show\_site\_numbers}

\end{itemize}

If present, numbers corresponding to the sites of the cell are drawn.
\begin{itemize}
\item {} 
\textbf{multicolor}  (default = false)

\end{itemize}

If present, each isomer is filled with different colors.
\begin{quote}

Multicolor: (Inspired on the Day Of The Tentacle)
\begin{quote}

Cis : Blue rectangles, pink circles

L-trans: Magenta rectangles, green circles

D-trans: Green rectangles, yellow circles
\end{quote}
\end{quote}
\begin{itemize}
\item {} 
\textbf{drawing\_frequency}  \code{frequency.int} (default = 100)

\end{itemize}

Frames will be printed after \code{frequency.int} KMC steps.
\begin{itemize}
\item {} 
\textbf{drawing\_scale}  \code{scale.int} (default = 800)

\end{itemize}

Set the scale of the drawings in the frames.
\begin{itemize}
\item {} 
\textbf{drawing\_title\_size}  \code{scale.int} (default = 20)

\end{itemize}

Set the size for drawing the title.
\begin{itemize}
\item {} 
\textbf{show\_images}  (default = false)

\end{itemize}

If present, files .fig  are created.
\begin{itemize}
\item {} 
\textbf{no\_jpgs}  (default = false)

\end{itemize}

If present, the program will not generate the jpgs during the execution.
After the simulation, jpgs images can be generated from .fig files
executing
\begin{quote}

\code{fig2dev -L jpeg frame.fig frame.jpg}
\end{quote}

If \textbf{show\_images} is not present and \textbf{no\_jpgs} is not present, then no frames
will be created.
\begin{itemize}
\item {} 
\textbf{write\_kmcout} {[}\textbf{formatted} (default = unformatted) {]} (default = false)

\end{itemize}

If present, kmc.out file will be written. By default it will be unformatted,
but if the optional \textbf{formatted} keyword is present, it will be written in a
human-readable form. This file contains the history of the entire simulation.

\emph{NOTE:} kmc.out file becomes rapidly huge.


\chapter{Indices and tables}
\label{index:indices-and-tables}\begin{itemize}
\item {} 
\emph{genindex}

\item {} 
\emph{modindex}

\item {} 
\emph{search}

\end{itemize}



\renewcommand{\indexname}{Index}
\printindex
\end{document}
